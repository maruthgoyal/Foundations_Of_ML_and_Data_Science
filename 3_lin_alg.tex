\section{Linear Algebra Review}
\subsection{Matrix Norms}
This section is heavily borrowed from the great Wikipedia article \cite{wiki:Matrix_norm}.
\begin{defn}[Matrix Norm]
A matrix norm $\norm{\cdot}: \R^{d \times n} \to \R$ is any real-valued function on matrices satisfying the following properties:
\begin{enumerate}
    \item $\norm{\alpha \A} = |\alpha| \norm{\A}$
    \item $\norm{\A + {\bf B}} \leq \norm{\A} + \norm{\bf B}$
    \item $\norm{\A} \geq 0$
    \item $\norm{\A} = 0 \iff \A = {\bf 0}$
\end{enumerate}
\end{defn}
\begin{prop}[Submultiplicative Norms]
    A norm $\norm{\cdot}$ is said to be {\em submultiplicative} if it satisfies
    \begin{align*}
        \norm{\A {\bf B}} \leq \norm{\A} \cdot \norm{\bf B}
    \end{align*}
\end{prop}

\begin{remark}
    The notation for matrix norms is {\em horrible}. $\norm{\A}_2$ can mean the $2$-operator norm, or the entry-wise $2$-norm, or the Schatten $2$-norm (Frobenius norm).
    Always make sure you know exactly which norm the authors mean, and if possible be explicit about it if/when you use the notation if not immedeately
    clear from context.
\end{remark}
\subsubsection{Vector-induced (Operator) norms}
\label{sec:linalg:opnorm}
Given any matrix $\A \in \R^{d \times n}$, and any vector norm $\norm{\cdot}: \R^d \to \R$, we define
the induced matrix norm to be
\begin{align*}
    \norm{\A} = \sup \, \set{\norm{\A \x} \, \mid \, \x \in \R^n \, , \, \norm{\x} = 1}
\end{align*}
   \paragraph{Special Cases}
   \begin{enumerate}
       \item When the norm is $\norm{\cdot}_1$, i.e. the $1$-norm for vectors, we have
       \begin{align*}
           \norm{\A}_1 = \max_{1 \leq j \leq n} \norm{\A_{\bullet, j}}_1
       \end{align*}
       i.e., the largest $1$-norm among the columns.
       \item When the norm is $\norm{\cdot}_2$, i.e. the $2$-norm for vectors, we have
       \begin{align*}
           \norm{\A}_2 = \sigma_{\max{}}(\A)
       \end{align*}
       i.e., the largest singular value of $\A$. For square matrices, equivalently this is the largest
       eigenvalue $\lambda_{\max{}}(\A)$. Also known as the {\em Spectral Norm}.
   \end{enumerate}
   \paragraph{Properties}
   \begin{enumerate}
       \item Any operator norm is {\em submultiplicative}.
   \end{enumerate}

\subsubsection{Entry-wise norms}
   Another way to use vector norms to create matrix norms is to simply flatten an $n \times d$ matrix into a vector
   of length $nd$, and then use a vector norm $\norm{\cdot}: \R^{dn} \to \R$. Thus the entry-wise norm under
   the $p$-vector norm is
   \begin{align*}
       \norm{A}_p = \lp \sum_{i,j} |a_{ij}|^p \rp^{1/p}
   \end{align*}
   \paragraph{Special Cases}
   \begin{enumerate}
       \item When $p = 2$, this is the Frobenius norm $\norm{\cdot}_F$.
       \item When $p = \infty$ this is the $\ell_\infty$ norm, i.e., simply the maximum value in the matrix.
   \end{enumerate}
\subsubsection{Schatten $p$-norms}
The Schatten $p$-norms are defined in terms of the singular values of the matrix $\A$. In particular,
if $\set{\sigma_i}_{i \in [r]}$ are the singular values for a matrix $\A$, then the Schatten $p$-norm is
\begin{align*}
    \norm{\A}_p = \lp \sum_{i \in [r]} \sigma_i^p \rp^{1/p}
\end{align*}

\paragraph{Special Cases}
\begin{enumerate}
    \item When $p = 1$, this is known as the {\em Nuclear Norm}.
    \item When $p = 2$, this is again equivalent to the {\em Frobenius Norm} $\norm{\cdot}_F$.
\end{enumerate}
\paragraph{Properties}
\begin{enumerate}
    \item All Schatten norms are {\em submultiplicative}
    \item The Schatten norms are invariant under unitary operations. i.e., $\norm{U\A V}_p = \norm{\A}_p$ where $U,V$ are unitaries.
\end{enumerate}

\subsection{Frobenius norm}
The Frobenius norm is a $2$-norm as both a Schatten $2$-norm, and an entry-wise norm. i.e.,
\begin{align*}
    \norm{\A}_F = \sqrt{\sum_{i \in [r]} \sigma_i^2} = \sqrt{\tr\lp \A^\top \A \rp} = \sqrt{\sum_{i,j} |a_{i,j}|^2}
\end{align*}
It is also in fact the norm induced by the {\em Frobenius Inner Product}
\begin{defn}[Frobenius Inner Product]
    Given $\A, {\bf B}$ the Frobenius inner product $\inn{\A, {\bf B}}$ is defined as
    \begin{align*}
        \inn{\A, {\bf B}}_F = \tr\lp \A^\top {\bf B} \rp 
    \end{align*}
\end{defn}
Thus we also get the following equality for the Frobenius norm
\begin{prop}
$\norm{A + B}_F^2 = \norm{A}_F^2 + \norm{B}_F^2 + 2\inn{A, B}_F$
\end{prop}
\begin{prop}
    $\norm{\A^\top\A}_F = \norm{\A\A^\top}_F \leq \norm{\A}_F^2$
\end{prop}
Of course, the norm also enjoys the properites of unitary invariance, and submultiplicity of Schatten norms.
\subsection{Spectral Decomposition of Matrices}
\begin{theorem}[Spectral Theorem]
    For a Hermitian operator $\A$ over $\C$ (i.e., $\A = \A^*$), there exists an orthogonal basis
    composed of its eigenvectors, and further each eigenvalue is real.
\end{theorem}
\begin{defn}[Positive Semidefinite]
    A symmetric matrix $\A \in \R^{n \times n}$ is said to be positive semidefinite iff
    \begin{align*}
        \x^\top \A \x \geq 0 \quad \forall \x \in \R^n
    \end{align*}
    If this inequality is strict, then it's said to be positive definite.
\end{defn}
\begin{example}
    For any rectangular matrix ${\bf X} \in \R^{d \times n}$, both $\X\X^\top$ and $\X^\top \X$ are positive semidefinite.
\end{example}
\begin{theorem}
    A symmetric matrix is positive definite (semidefinite) iff all of its eigenvalues are positive (nonnegative).
\end{theorem}
\subsection{Singular Value Decomposition}
\subsubsection{Principal Components Analysis}
\subsubsection{Power Method}
\subsubsection{QR Decomposition}